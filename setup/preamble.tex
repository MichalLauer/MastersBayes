\usepackage[top=25mm,bottom=25mm,right=25mm,left=30mm,head=12.5mm,foot=12.5mm]{geometry}
\let\openright=\clearpage

%%% Pokud tiskneme oboustranně:
%\documentclass[11pt,a4paper,twoside,openright]{report}
%\usepackage[top=25mm,bottom=25mm,right=25mm,left=30mm,head=12.5mm,foot=12.5mm]{geometry}
%\let\openright=\cleardoublepage

%%% DEFINICE ZÁKLADNÍCH PROMĚNNÝCH
% \def\TypPrace{BP}                % bakalářská práce/bachelor thesis
\def\TypPrace{DP}               % diplomová práce/master thesis
\def\Jazyk{eng}                  % čeština/czech

%%% Definice různých užitečných maker (viz popis uvnitř souboru)
%%% Tento soubor obsahuje definice různých užitečných maker a prostředí %%%
%%% Další makra připisujte sem, ať nepřekáží v ostatních souborech.     %%%
%%% This file contains definitions of various useful macros and environments      %%%
%%% Assign additional macros here so that they do not interfere with other files. %%%

\usepackage{ifpdf}
\usepackage{ifxetex}
\usepackage{ifluatex}

%%% Nastavení pro použití samostatné bibliografické databáze.
%%% Settings for using a separate bibliographic database.
\usepackage[
   backend=biber
  ,style=apa
  % ,citestyle=numeric-comp
  ,sortlocale=cs_CZ
  ,alldates=iso
  ,bibencoding=UTF8
  ,maxnames=2
  ,maxbibnames=99
  % ,block=ragged
]{biblatex}
\let\cite\parencite
%\renewcommand*{\multinamedelim}{, \addspace}             % ISO-690
%\renewcommand*{\finalnamedelim}{\addspace a \addspace}   % ISO-690

\bibliography{./literatura.bib}

%\makeatletter
%\RequireBibliographyStyle{numeric}
%\makeatother

\makeatletter
\newcommand{\apamaxcitenames}{1}

\DeclareNameFormat{labelname}{%
  % First set the truncation point
  \ifthenelse{\value{uniquelist}>1}
    {\numdef\cbx@min{\value{uniquelist}}}
    {\numdef\cbx@min{\value{minnames}}}%
  % Always print the first name and the second if there are only two since
  % "et al" must always be plural
  \ifboolexpr{test {\ifnumcomp{\value{listcount}}{=}{1}}
              or test {\ifnumcomp{\value{listtotal}}{=}{2}}}
    {\usebibmacro{labelname:doname}%
      {\namepartfamily}%
      {\namepartfamilyi}%
      {\namepartgiven}%
      {\namepartgiveni}%
      {\namepartprefix}%
      {\namepartprefixi}%
      {\namepartsuffix}%
      {\namepartsuffixi}}
    % We are looking at name >=3
    % If the list is 6 or more names or we have seen citation before, potential truncation
    {\ifboolexpr{test {\ifnumcomp{\value{listtotal}}{>}{\apamaxcitenames}}
                 or test {\ifciteseen}}
     % Less than the truncation point, print normally
     {\ifnumcomp{\value{listcount}}{<}{\cbx@min + 1}
       {\usebibmacro{labelname:doname}%
         {\namepartfamily}%
         {\namepartfamilyi}%
         {\namepartgiven}%
         {\namepartgiveni}%
         {\namepartprefix}%
         {\namepartprefixi}%
         {\namepartsuffix}%
         {\namepartsuffixi}}
       {}%
      % At potential truncation point ...
      \ifnumcomp{\value{listcount}}{=}{\cbx@min + 1}
        % but enforce plurality of et al - only truncate here if there is at
        % least one more element after the current potential truncation point
        % so that "et al" covers at least two elements.
        {\ifnumcomp{\value{listcount}}{<}{\value{listtotal}}
          {\printdelim{andothersdelim}\bibstring{andothers}}
          {\usebibmacro{labelname:doname}%
            {\namepartfamily}%
            {\namepartfamilyi}%
            {\namepartgiven}%
            {\namepartgiveni}%
            {\namepartprefix}%
            {\namepartprefixi}%
            {\namepartsuffix}%
            {\namepartsuffixi}}}
        {}%
      % After truncation point, do not print name
      \ifnumcomp{\value{listcount}}{>}{\cbx@min + 1}
       {\relax}%
       {}}%
     % We are looking at name >=3
     % Name list is < 6 names or we haven't seen this citation before, print normally
     {\usebibmacro{labelname:doname}%
       {\namepartfamily}%
       {\namepartfamilyi}%
       {\namepartgiven}%
       {\namepartgiveni}%
       {\namepartprefix}%
       {\namepartprefixi}%
       {\namepartsuffix}%
       {\namepartsuffixi}}}}
\makeatother


%% Přepneme na českou sazbu, fonty Latin Modern a kódování češtiny
\ifthenelse{\boolean{xetex}\OR\boolean{luatex}}
   { % use fontspec and OpenType fonts with utf8 engines
			\usepackage[autostyle,english=british,czech=quotes]{csquotes}
			\usepackage{fontspec}
			\defaultfontfeatures{Ligatures=TeX,Scale=MatchLowercase}
   }
   {
			\usepackage{lmodern}
			\usepackage[T1]{fontenc}
			\usepackage{textcomp}
			\usepackage[utf8]{inputenc}
			\usepackage[autostyle,english=british,czech=quotes]{csquotes}
	 }
\ifluatex
\makeatletter
\let\pdfstrcmp\pdf@strcmp
\makeatother
\fi

\usepackage[a-2u]{pdfx}     % výsledné PDF bude ve standardu PDF/A-2u
                            % resulting PDF will be in the PDF / A-2u standard

%%% Další užitečné balíčky (jsou součástí běžných distribucí LaTeXu)
\usepackage{amsmath}        % rozšíření pro sazbu matematiky / extension for math typesetting
\usepackage{amsfonts}       % matematické fonty / mathematical fonts
\usepackage{amssymb}        % symboly / symbols
\usepackage{amsthm}         % sazba vět, definic apod. / typesetting of sentences, definitions, etc.
\usepackage{bm}             % tučné symboly (příkaz \bm) / bold symbols (\bm command)
\usepackage{graphicx}       % vkládání obrázků / graphics inserting
\usepackage{listings}       % vylepšené prostředí pro strojové písmo / improved environment for source codes typesetting
\usepackage{fancyhdr}       % prostředí pohodlnější nastavení hlavy a paty stránek / environment for more comfortable adjustment of the head and foot of the pages
\usepackage{icomma}         % inteligetní čárka v matematickém módu / intelligent comma in math mode
\usepackage{dcolumn}        % lepší zarovnání sloupců v tabulkách / better alignment of columns in tables
\usepackage{booktabs}       % lepší vodorovné linky v tabulkách / better horizontal lines in tables
\usepackage{tabularx}       % vhodné pro tabulky s delšími texty / suitable for tables with longer texts
\makeatletter
\@ifpackageloaded{xcolor}{
   \@ifpackagewith{xcolor}{usenames}{}{\PassOptionsToPackage{usenames}{xcolor}}
  }{\usepackage[usenames]{xcolor}} % barevná sazba / color typesetting
\makeatother
\usepackage{multicol}       % práce s více sloupci na stránce / work with multiple columns on a page
\usepackage{caption}
\usepackage{enumitem}
\setlist[itemize]{noitemsep, topsep=0pt, partopsep=0pt}
\setlist[enumerate]{noitemsep, topsep=0pt, partopsep=0pt}
\setlist[description]{noitemsep, topsep=0pt, partopsep=0pt}

\usepackage{tocloft}
\setlength\cftparskip{0pt}
\setlength\cftbeforechapskip{1.5ex}
\setlength\cftfigindent{0pt}
\setlength\cfttabindent{0pt}
\setlength\cftbeforeloftitleskip{0pt}
\setlength\cftbeforelottitleskip{0pt}
\setlength\cftbeforetoctitleskip{0pt}
\renewcommand{\cftlottitlefont}{\Huge\bfseries\sffamily}
\renewcommand{\cftloftitlefont}{\Huge\bfseries\sffamily}
\renewcommand{\cfttoctitlefont}{\Huge\bfseries\sffamily}

% vyznaceni odstavcu
% differentiation of new paragraphs
\parindent=0pt
\parskip=11pt

% zakaz vdov a sirotku - jednoradkovych pocatku ci koncu odstavcu na prechodu mezi strankami
% Prohibition of widows and orphans - single-line beginning and end of paragraph at the transition between pages
\clubpenalty=1000
\widowpenalty=1000
\displaywidowpenalty=1000

% nastaveni radkovani
% setting of line spacing
\renewcommand{\baselinestretch}{1.20}

% nastaveni pro nadpisy - tucne a bezpatkove
% settings for headings - bold and sans serif
\usepackage{sectsty}
\allsectionsfont{\sffamily}

% nastavení hlavy a paty stránek
% page head and foot settings
\makeatletter
\if@twoside%
    \fancypagestyle{fancyx}{%
			\fancyhf{}
      \fancyhead[RE]{\rightmark}
      \fancyhead[LO]{\leftmark}
      \fancyfoot[RO,LE]{\thepage}
      \renewcommand{\headrulewidth}{.5pt}
      \renewcommand{\footrulewidth}{.5pt}
    }
    \fancypagestyle{plain}{%
			\fancyhf{}
    	\fancyfoot[RO,LE]{\thepage}
    	\renewcommand{\headrulewidth}{0pt}
    	\renewcommand{\footrulewidth}{0.5pt}
    }
\else
    \fancypagestyle{fancyx}{%
			\fancyhf{}
      \fancyhead[R]{\leftmark}
      \fancyfoot[R]{\thepage}
      \renewcommand{\headrulewidth}{.5pt}
      \renewcommand{\footrulewidth}{.5pt}
    }
    \fancypagestyle{plain}{%
    	\fancyhf{} % clear all header and footer fields
    	\fancyfoot[R]{\thepage}
    	\renewcommand{\headrulewidth}{0pt}
    	\renewcommand{\footrulewidth}{0.5pt}
    }
\fi
\renewcommand*{\cleardoublepage}{\clearpage\if@twoside \ifodd\c@page\else
	\hbox{}%
	\thispagestyle{empty}%
	\newpage%
	\if@twocolumn\hbox{}\newpage\fi\fi\fi
}
\makeatother

% Tato makra přesvědčují mírně ošklivým trikem LaTeX, aby hlavičky kapitol
% sázel příčetněji a nevynechával nad nimi spoustu místa. Směle ignorujte.
% These macros convince with a slightly ugly LaTeX trick to make chapter headers
% bet more sane and didn't miss a lot of space above them. Be boldly ignore it.
\makeatletter
\def\@makechapterhead#1{
  {\parindent \z@ \raggedright \sffamily
   \Huge\bfseries \thechapter. #1
   \par\nobreak
   \vskip 20\p@
}}
\def\@makeschapterhead#1{
  {\parindent \z@ \raggedright \sffamily
   \Huge\bfseries #1
   \par\nobreak
   \vskip 20\p@
}}
\makeatother

% Trochu volnější nastavení dělení slov, než je default.
% Slightly looser hyphenation setting than default.
\lefthyphenmin=2
\righthyphenmin=2

% Zapne černé "slimáky" na koncích řádků, které přetekly, abychom si jich lépe všimli.
% Turns on the black "snails" at the ends of the lines that overflowed to get us noticed them better.
\overfullrule=1mm

\def\BibTeX{{\rm B\kern-.05em{\sc i\kern-.025em b}\kern-.08em
    T\kern-.1667em\lower.7ex\hbox{E}\kern-.125emX}}

%% Balíček hyperref, kterým jdou vyrábět klikací odkazy v PDF,
%% ale hlavně ho používáme k uložení metadat do PDF (včetně obsahu).
%% Většinu nastavítek přednastaví balíček pdfx.
%% A hyperref package that can be used to produce clickable links in PDF,
%% but we mainly use it to store metadata in PDF (including content).
%% Most settings are preset by the pdfx package.
\hypersetup{unicode}
\hypersetup{breaklinks=true}
\hypersetup{hidelinks}
\hypersetup{colorlinks=true,urlcolor=blue}

\renewcommand{\UrlBreaks}{\do\/\do\=\do\+\do\-\do\_\do\ \do\a\do\b\do\c\do\d%
\do\e\do\f\do\g\do\h\do\i\do\j\do\k\do\l\do\m\do\n\do\o\do\p\do\q\do\r\do\s%
\do\t\do\u\do\v\do\w\do\x\do\y\do\z\do\A\do\B\do\C\do\D\do\E\do\F\do\G\do\H%
\do\I\do\J\do\K\do\L\do\M\do\N\do\O\do\P\do\Q\do\R\do\S\do\T\do\U\do\V\do\W%
\do\X\do\Y\do\Z\do\1\do\2\do\3\do\4\do\5\do\6\do\7\do\8\do\9\do\0}
\urlstyle{tt}

%%% Prostředí pro sazbu kódu, případně vstupu/výstupu počítačových
%%% programů. (Vyžaduje balíček listings -- fancy verbatim.)
%%% Environment for source code typesetting, or computer input/output
%%% programs. (Requires package listings - fancy verbatim.)
\setmonofont[
  Path = ./fonts/,
  Extension = .ttf,
  UprightFont = *-Retina,
  BoldFont = *-Bold,
  LightFont = *-Light,
  MediumFont = *-Medium,
  SemiBoldFont = *-SemiBold
]{FiraCode}

\lstnewenvironment{code}[3]{
  \vspace{2em}
  \lstset{
    language=#1,
    caption=#2,
    label=#3,                          % the language of the code
    basicstyle=\ttfamily,              % the size of the fonts that are used for the code
    numbers=left,                      % where to put the line-numbers
    numberstyle=\color{Blue},          % the style that is used for the line-numbers
    stepnumber=1,                      % the step between two line-numbers. If it is 1, each line
    numbersep=5pt,                     % how far the line-numbers are from the code
    backgroundcolor=\color{white},     % choose the background color. You must add \usepackage{color}
    showspaces=false,                  % show spaces adding particular underscores
    showstringspaces=false,            % underline spaces within strings
    showtabs=false,                    % show tabs within strings adding particular underscores
    frame=single,                      % adds a frame around the code
    rulecolor=\color{black},           % if not set, the frame-color may be changed on line-breaks within not-black text (e.g. commens (green here))
    tabsize=2,                         % sets default tabsize to 2 spaces
    captionpos=b,                      % sets the caption-position to bottom
    breaklines=true,                   % sets automatic line breaking
    breakatwhitespace=false,           % sets if automatic breaks should only happen at whitespace
    keywordstyle=\color{RoyalBlue},    % keyword style
    commentstyle=\color{YellowGreen},  % comment style
    stringstyle=\color{ForestGreen},   % string literal style
    otherkeywords={set.seed,binom.test,file.path,R2WinBUGS::bugs,rjags::jags.model,
    rjags::update.jags,coda::coda.samples,seq_along,coda::effectiveSize,rstan::stan,
    rowwise,mutate,get_ab,case_when,ungroup,bind_rows,tibble,glue,seq_len},
    deletekeywords={q,lower,upper,data,_,model.file,save,beta,path,file,model,
    variable.names,var,by}
  }
}{}

\lstnewenvironment{model}[3]{
  \vspace{2em}
  \lstset{
    language=#1,
    caption=#2,
    label=#3,                          % the language of the code
    basicstyle=\ttfamily,              % the size of the fonts that are used for the code
    numbers=left,                      % where to put the line-numbers
    numberstyle=\color{Blue},          % the style that is used for the line-numbers
    stepnumber=1,                      % the step between two line-numbers. If it is 1, each line
                                       % will be numbered
    numbersep=5pt,                     % how far the line-numbers are from the code
    backgroundcolor=\color{white},     % choose the background color. You must add \usepackage{color}
    showspaces=false,                  % show spaces adding particular underscores
    showstringspaces=false,            % underline spaces within strings
    showtabs=false,                    % show tabs within strings adding particular underscores
    frame=single,                      % adds a frame around the code
    rulecolor=\color{black},           % if not set, the frame-color may be changed on line-breaks within not-black text (e.g. commens (green here))
    tabsize=2,                         % sets default tabsize to 2 spaces
    captionpos=b,                      % sets the caption-position to bottom
    breaklines=true,                   % sets automatic line breaking
    breakatwhitespace=false,           % sets if automatic breaks should only happen at whitespace
    deletekeywords={binomial,data,sample,is,equal,lower,upper,attr,_,model,for,dbeta,beta,mean,sd,eff,deviance,
    For,scale,factor,variable,real,se_,diag_,__,on,split,lp__,diag_}
  }
}{}

%%% Tato část obsahuje texty závislé na typu práce, jazyku a pohlaví %%%
%%% This part contains texts depending on the type of work, language and gender %%%

\newcommand{\ifstringequal}[4]{%
  \ifnum\pdfstrcmp{#1}{#2}=0
  #3%
  \else
  #4%
  \fi
}

\def\TypPraceBP{BAKALÁŘSKÁ PRÁCE}
\def\TypPraceDP{DIPLOMOVÁ PRÁCE}
\def\SeznamZkratek{Seznam použitých zkratek}
\def\Prilohy{Přílohy}
\def\VSE{Vysoká škola ekonomická v Praze}
\def\FIS{Fakulta informatiky a statistiky}
\def\StudijniProgramText{Studijní program}
\def\SpecializaceText{Specializace}
\def\AutorText{Autor}
\def\VedouciText{Vedoucí práce}
\def\KonzultantText{Konzultant práce}
\def\Praha{Praha}
\def\PodekovaniText{Poděkování}
\def\bibnamex{Použitá literatura}
\def\bibnamey{Použité balíčky}
\renewcommand*{\lstlistlistingname}{Seznam zdrojových kódů}
\def\lstlistingname{Výpis}

% Ovládá se pomocí parametru 'lang'
% \makeatletter
% \ifstringequal{\Jazyk}{eng}{\main@language{english}}{}
% \ifstringequal{\Jazyk}{slo}{\main@language{slovak}}{}
% \makeatother

\ifstringequal{\Jazyk}{eng}{
         \def\TypPraceBP{BACHELOR THESIS}
         \def\TypPraceDP{MASTER THESIS}
				 \def\SeznamZkratek{List of abbreviations}
				 \def\Prilohy{Appendices}
				 \def\VSE{Prague University of Economics and Business}
				 \def\FIS{Faculty of Informatics and Statistics}
				 \def\StudijniProgramText{Study program}
				 \def\SpecializaceText{Specialization}
				 \def\AutorText{Author}
				 \def\VedouciText{Supervisor}
         \def\KonzultantText{Consultant}
				 \def\Praha{Prague}
				 \ifstringequal{\TypPrace}{BP}{\def\Praci{bachelor thesis }}{}
         \ifstringequal{\TypPrace}{DP}{\def\Praci{master thesis }}{}
				 \def\PodekovaniText{Acknowledgements}
				 \def\bibnamex{References}
         \renewcommand*{\lstlistlistingname}{List of source codes}
         \def\lstlistingname{Source code}
				 \ifdef{\finalnamedelim}{\renewcommand*{\finalnamedelim}{\addspace and \addspace}}%
}{}

\ifstringequal{\Jazyk}{slo}{
         \def\TypPraceBP{BAKALÁRSKA PRÁCA}
         \def\TypPraceDP{DIPLOMOVÁ PRÁCA}
				 \def\SeznamZkratek{Zoznam použitých skratiek}
				 \def\Prilohy{Prílohy}
				 \def\StudijniProgramText{Študijný program}
				 \def\SpecializaceText{Špecializácia}
				 \def\VedouciText{Vedúci práce}
         \def\KonzultantText{Konzultant práce}
				 \ifstringequal{\TypPrace}{BP}{\def\Praci{bakalársku prácu }}{}
         \ifstringequal{\TypPrace}{DP}{\def\Praci{diplomovú prácu }}{}
				 \def\PodekovaniText{Poďakovanie}
				 \def\bibnamex{Použitá literatúra}
				 \renewcommand*{\lstlistlistingname}{Zoznam zdrojových kódov}
				 \def\lstlistingname{Výpis}
}{}

\ifstringequal{\TypPrace}{BP}{\def\TypPraceText{\TypPraceBP}}{}
\ifstringequal{\TypPrace}{DP}{\def\TypPraceText{\TypPraceDP}}{}


%%% Název práce v jazyce práce (přesně podle zadání)
%%% Title of the thesis in the language used in the text (exact according to assignment)
\def\NazevPrace{Bayesian analysis of time-varying volatility models}

%%% Jméno autora
%%% Author's name - First name Surname
\def\AutorPrace{Bc. et Bc. Michal Lauer}

%%% Rok odevzdání a měsíc (slovně)
%%% Year of submission and month (verbally) - month YYYY
\def\DatumOdevzdani{June 2025}

%%% Vedoucí práce: Jméno a příjmení s~tituly
%%% Supervisor: First name and surname with titles
\def\Vedouci{Ing. Miroslav Plašil, Ph.D.}

% %%% Konzultant práce: Jméno a příjmení s~tituly
% %%% Consultant: First name and surname with titles
\def\Konzultant{}

%%% Studijní program
%%% Study program
\def\StudijniProgram{Statistics}

%%% Studijní program - specializace
%%% Study program - specialization
\def\Specializace{Econometrics}

%%% Nepovinné poděkování (vedoucímu práce, konzultantovi, tomu, kdo zapůjčil software, literaturu apod.)
%%% Optional thanks (the supervisor, the consultant, the borrower of software, literature, etc.)
\def\Podekovani{%
I would like to thank my supervisor for their guidance in the world of bayesian statistics.
}

%%% Abstrakt (doporučený rozsah cca 150-250 slov; nejedná se o zadání práce)
\def\Abstrakt{

Diplomová práce se zabývá modelováním volatility finančních časových řad. V teoretické části jsou představeny různé
transformace, přičemž jako nejvhodnější forma pro analýzu volatility je zvolena logaritmická návratnost. Dále jsou
popsány modely podmíněné heteroskedasticity, konkrétně ARCH, GARCH a stochastické volatilní (SV) modely, a to jak po
teoretické, tak po praktické stránce. Následuje úvod do bayesovského přístupu, který je využit pro odhad uvedených
modelů pomocí simulačních technik, zejména metod generujících vzorky z posteriorního rozdělení. Závěrem je uveden
přehled simulačních přístupů k tvorbě posteriorních vzorků a propojení těchto modelů s bayesovským rámcem
prostřednictvím dostupné literatury.

V praktické části jsou analyzovány dvě finanční časové řady s projevy heteroskedasticity a shluků volatility. První
řada je modelována pomocí ARCH, GARCH a SV modelů, přičemž výsledky ukazují, že SV model lépe vystihuje volatilní
strukturu a dokáže se lépe přizpůsobit náhlým změnám. Na základě těchto výsledků je druhá časová řada modelována pouze
pomocí SV modelu. Posteriorní rozdělení parametrů jsou mezi řadami podobná, avšak latentní volatilita se liší, což
naznačuje odlišnou strukturu rizika. Výstupem práce jsou rovněž obecné skripty pro odhad uvedených modelů v nástroji
Stan.

}

\def\AbstraktEN{%

This thesis focuses on modeling the volatility of financial time series. The theoretical part introduces various data
transformations, selecting logarithmic returns as the most suitable form for volatility analysis. Subsequently, models
of conditional heteroskedasticity are presented—specifically ARCH, GARCH, and stochastic volatility (SV) models — both
from theoretical and practical perspectives. An introduction to the Bayesian approach follows, which is employed for
estimating the aforementioned models using simulation-based techniques, particularly methods for generating samples
from posterior distributions. The section concludes with an overview of simulation approaches for posterior sampling
and the integration of these models within the Bayesian framework through a review of relevant literature.

The empirical part analyzes two financial time series exhibiting signs of heteroskedasticity and volatility clustering.
The first series is modeled using ARCH, GARCH, and SV models, with results indicating that the SV model more accurately
captures the volatility structure and better adapts to sudden changes. Based on these findings, the second time series
is modeled exclusively using the SV model. While posterior parameter distributions are similar across both series, the
latent volatility differs, suggesting a distinct risk structure. The thesis also provides general-purpose scripts for
estimating the discussed models using the Stan platform.

}

%%% 3 až 5 klíčových slov (doporučeno)
\def\KlicovaSlova{Bayesovská statistika, Finanční časové řady, ARCH, GARCH, SV}
\def\KlicovaSlovaEN{Bayesian statistics, Financial time series, ARCH, GARCH, SV}

% Quarto fixes
\setcounter{secnumdepth}{2}
\usepackage[output-decimal-marker={.}]{siunitx}
